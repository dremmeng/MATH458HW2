% LaTeX Article Template - customizing page format
%
% LaTeX document uses 10-point fonts by default.  To use
% 11-point or 12-point fonts, use \documentclass[11pt]{article}
% or \documentclass[12pt]{article}.
\documentclass{article}

% Set left margin - The default is 1 inch, so the following 
% command sets a 1.25-inch left margin.
\setlength{\oddsidemargin}{0.25in}

% Set width of the text - What is left will be the right margin.
% In this case, right margin is 8.5in - 1.25in - 6in = 1.25in.
\setlength{\textwidth}{6in}

% Set top margin - The default is 1 inch, so the following 
% command sets a 0.75-inch top margin.
\setlength{\topmargin}{-0.25in}

% Set height of the text - What is left will be the bottom margin.
% In this case, bottom margin is 11in - 0.75in - 9.5in = 0.75in
\setlength{\textheight}{8in}
\usepackage{fancyhdr}
\usepackage{float}
\usepackage{mathtools}
\usepackage{amsmath}
\usepackage{amssymb}
\usepackage{graphicx}
\usepackage{float}
\graphicspath{ {./} }
\setlength{\parskip}{5pt} 
\pagestyle{fancyplain}
% Set the beginning of a LaTeX document
\begin{document}

\lhead{Drew Remmenga MATH 458}
\rhead{HW \#2}
%\lhead{Independent Study}
%\rhead{R Lab}

\begin{enumerate}

\item 
	\begin{enumerate}
	\item Associativity. Since $H$ and $K$ are subgroups they are associative so $H\cap K$ inherits associativity.
	\item Identity. As subgroups of $G$ we can assume $e$, the identity of $G$, is $\in H$ and $\in K$.
	\item Closure. $\forall a,b \in H$ we can say $ab \in H$ and $\forall a,b \in K$ we can say $ab \in K$. So $\forall a,b \in H\cap K$ we can say $ab \in H\cap K$.
	\item Inverses. Since $H$ and $K$ are subgroups we can say $\exists c^{-1} \in H$ and $K$ for $\forall c \in H$ and $K$. So $\exists c^{-1} \in H \cap K$ for $ \forall c \in H\cap K$.
	\end{enumerate}
\item For the finite case we have: $(x a x^{-1})^{n} = x a^{n}x^{-1}$ so substituting $| a |$ in for $n$ leaves us with $x a ^{| a |} x ^{-1} = x e x^{-1} = e$. For the infinite case we have $(x a x^{-1})^{n} = x a^{n}x^{-1}$ so if the order of $a$ is infinite the order of $x a x^{-1}$ cannot be finite. 
\item Consider $H = {e,a,b,ab}$ with $|a|, |b| = 2$ for $a,b \in G$. Then $H$ is a valid subgroup of order 4 and is a subgroup of $G$.
\item Let $H$ be the smallest subgroup of $G$ containing $a$. Then by closure $\forall a^{n}$ for $\forall n \in \mathbb{Z}$ is $\in H$. Then $\langle a \rangle \subset H$ and since $H$ is the smallest subgroup of $G$ containing $a$ then $\langle a \rangle$ is the smallest subset of $G$ containing $a$. 
\item
	\begin{enumerate}
	\item Since $G$ is abelian the centeralizer of the group is equal to $G$. Since $G$ is abelian the center of an element $a$ is equal to the group $G$ for $\forall a \in G$.
	\item  Yes since the center commutes with all other elements it is abelian.
	\item The centralizer of a group element need not be abelian since that element may not be abelian with all other elements in the centralizer of that element. 
	\end{enumerate}
\end{enumerate}




\end{document}